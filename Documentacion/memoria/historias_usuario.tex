\phantomsection
\section*{Historias de Usuario}
\addcontentsline{toc}{section}{Historias de Usuario}

%\phantomsection
\subsection*{HU1: Registro de nuevo usuario}
%\addcontentsline{toc}{subsection}{HU1: Registro de nuevo usuario}
\begin{description}
  \item[Como:] usuario nuevo
  \item[quiero:] poder registrarme con mi número  de telefono y una contraseña
  \item[para:] poder acceder a la aplicación de mensajería
\end{description}

\textbf{Criterios de aceptación:}
\begin{itemize}
    \item El usuario puede introducir un número valido y una contraseña segura.
    \item El sistema valida el número y muestra errores claros si es incorrecto.
    \item El sistema informa si el número ya está registrado.
\end{itemize}

%\phantomsection
\subsection*{HU2: Inicio de sesión}
%\addcontentsline{toc}{subsection}{HU2: Inicio de sesión}
\begin{description}
  \item[Como:] usuario registrado
  \item[quiero:] iniciar sesión con mis credenciales
  \item[para:] acceder a mis contactos y conversaciones
\end{description}

\textbf{Criterios de aceptación:}
\begin{itemize}
    \item El usuario puede ingresar su número y contraseña correctamente.
    \item Si las credenciales son inválidas, el sistema muestra un mensaje de error.
    \item El acceso es denegado hasta que las credenciales sean válidas.
\end{itemize}

%\phantomsection
\subsection*{HU3: Añadir contacto}
%\addcontentsline{toc}{subsection}{HU3: Añadir contacto}
\begin{description}
  \item[Como:] usuario registrado
  \item[quiero:] poder añadir nuevos contactos mediante su identificador
  \item[para:] poder comunicarme con ellos desde la aplicación
\end{description}

\textbf{Criterios de aceptación:}
\begin{itemize}
    \item El sistema permite buscar usuarios por identificador.
    \item Se muestra un mensaje si el usuario no existe.
    \item El contacto se añade a la lista del usuario tras la confirmación.
\end{itemize}

%\phantomsection
\subsection*{HU4: Crear grupo}
%\addcontentsline{toc}{subsection}{HU4: Crear grupo}
\begin{description}
  \item[Como:] usuario registrado
  \item[quiero:] poder crear grupos de conversación
  \item[para:] mantener conversaciones con múltiples contactos al mismo tiempo
\end{description}

\textbf{Criterios de aceptación:}
\begin{itemize}
    \item El usuario puede nombrar el grupo y seleccionar contactos.
    \item El grupo se guarda con la información y miembros indicados.
    \item Se puede ver el grupo creado en la lista de grupos.
\end{itemize}

%\phantomsection
\subsection*{HU5: Enviar mensaje}
%\addcontentsline{toc}{subsection}{HU5: Enviar mensaje}
\begin{description}
  \item[Como:] usuario registrado
  \item[quiero:] poder enviar mensajes de texto a mis contactos o grupos
  \item[para:] comunicarme con mis contactos.
\end{description}

\textbf{Criterios de aceptación:}
\begin{itemize}
    \item El usuario puede escribir y enviar mensajes.
    \item Los mensajes se muestran inmediatamente en el chat.
    \item Los destinatarios reciben el mensaje cuando se conectan.
\end{itemize}

%\phantomsection
\subsection*{HU6: Ver historial de conversaciones}
%\addcontentsline{toc}{subsection}{HU6: Ver historial de conversaciones}
\begin{description}
  \item[Como:] usuario registrado
  \item[quiero:] poder ver el historial de mensajes de mis conversaciones
  \item[para:] seguir el contexto de la conversación
\end{description}

\textbf{Criterios de aceptación:}
\begin{itemize}
    \item El historial de cada chat se carga al abrirlo.
    \item El usuario puede desplazarse y leer mensajes anteriores.
    \item Los mensajes aparecen en orden cronológico.
\end{itemize}

%\phantomsection
\subsection*{HU7: Gestionar Contactos y Grupos}
%\addcontentsline{toc}{subsection}{HU7: Gestionar Contactos y Grupos}
\begin{description}
  \item[Como:] usuario registrado
  \item[quiero:] poder gestionar mis contactos y grupos
  \item[para:] mantener mi lista de contactos y grupos actualizada
\end{description}

\textbf{Criterios de aceptación:}
\begin{itemize}
    \item El usuario puede acceder a un formulario de edición.
    \item Los cambios se guardan y se reflejan inmediatamente.
    \item El sistema valida los campos antes de guardar.
\end{itemize}

%\phantomsection
\subsection*{HU8: Cerrar sesión}
%\addcontentsline{toc}{subsection}{HU8: Cerrar sesión}
\begin{description}
  \item[Como:] usuario registrado
  \item[quiero:] poder cerrar sesión de forma segura
  \item[para:] proteger el acceso a mi cuenta
\end{description}

\textbf{Criterios de aceptación:}
\begin{itemize}
    \item Hay una opción accesible para cerrar sesión.
    \item Al cerrar sesión se redirige al usuario a la pantalla de inicio.
    \item No se puede acceder a funciones privadas tras cerrar sesión.
\end{itemize}

%\phantomsection
\subsection*{HU9 - Convertirse en usuario premium}
%\addcontentsline{toc}{subsection}{HU9 - Convertirse en usuario premium}
\begin{description}
    \item[Como:]usuario registrado
    \item[quiero:] poder convertirme en usuario premium pagando una suscripción
    \item[para:] acceder a funciones adicionales como la exportación de mensajes
\end{description}

\textbf{Criterios de verificación:}
\begin{itemize}
    \item El sistema debe permitir al usuario registrarse como premium mediante el pago de una suscripción anual.
    \item El usuario premium debe tener acceso a funciones adicionales como la exportación de mensajes en PDF.
\end{itemize}

%\phantomsection
\subsection*{HU10 - Buscar mensajes}
%\addcontentsline{toc}{subsection}{HU10 - Buscar mensajes}
\begin{description}
    \item[Como:] usuario registrado
    \item[quiero:] buscar mensajes por fragmento de texto, nombre de contacto o número de teléfono
    \item[para:] encontrar fácilmente los mensajes que necesito
\end{description}

\textbf{Criterios de verificación:}
\begin{itemize}
    \item El sistema debe permitir buscar mensajes enviados o recibidos por el usuario filtrando por fragmento de texto, nombre del contacto o número de teléfono.
    \item Los resultados de la búsqueda deben mostrarse en forma de lista.
    \item El sistema debe permitir combinar varios criterios de búsqueda (por ejemplo, texto y contacto).
\end{itemize}

%\phantomsection
\subsection*{HU11 - Exportar mensajes a PDF (solo premium)}
%\addcontentsline{toc}{subsection}{HU11 - Exportar mensajes a PDF (solo premium)}
\begin{description}
    \item[Como:] usuario premium
    \item[quiero:] exportar mis conversaciones a un archivo PDF
    \item[para:]tener un registro de los mensajes intercambiados
\end{description}

\textbf{Criterios de verificación:}
\begin{itemize}
    \item El sistema debe permitir al usuario exportar sus conversaciones a PDF.
    \item El archivo PDF debe incluir los nombres de los participantes, el contenido del mensaje, y la fecha y hora de cada mensaje.
    \item El archivo PDF debe generarse correctamente y descargarse en el dispositivo del usuario.
\end{itemize}

\vspace{5ex}